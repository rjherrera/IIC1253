\documentclass[letter]{article}

\usepackage{MD_estilo}
\DeclareMathSymbol{\mrq}{\mathord}{operators}{`'}

\nombre{Raimundo Herrera Sufán} % Aqui va el nombre del alumno
\numtarea{6} % Aqui va el número de la tarea


\begin{document}

	\begin{pregunta}{1} % Aqui va el número de la pregunta
		\begin{enumerate}
% Pregunta 1.1
		\item 
		\underline{PD} Si $a$ a tiene un inverso multiplicativo en $\mathbb{Z}_n$, entonces gcd$(a,n)=1$.\\
		Si $a$ tiene un inverso multiplicativo, entonces existe un $a^{-1}$ tal que $a^{-1}a \equiv 1$ mod n.\\
		Ahora, partiendo de eso, se tiene, por la definición de módulo:
		\begin{align*}
			&a^{-1}a \equiv 1 \ \text{mod} \ n \\
			&a^{-1}a\  \text{mod} \ n = 1
		\end{align*}
		Luego, existe un único par $q, r$ en $\mathbb{Z}$, con $r=a^{-1}a$, tal que
		\begin{align*}
			&nq + 1 = a^{-1}a\\
			&a^{-1}a - nq = 1\\
			&(a^{-1})a + (-q)n = 1
		\end{align*}
		Ahora, si se toma $s=a^{-1}$ y $t=-q$, se tiene
		$$sa + tn = 1$$
		Teniendo lo anterior, junto con identidad de Bézout, que señala que el gcd entre dos números $a$ y $b$ es el mínimo número positivo tal que $sa+tb=\text{gcd}(a,b)$, se tiene que 1 es el máximo comun divisor entre $a$ y $n$. Esto, porque al encontrar una combinación lineal de $a$ y $n$ que lleve a 1, se tiene que, $0 < \text{gcd}(a, n) \leq 1$, ya que el gcd debe ser positivo.\\
		De ahí como el único entero mayor que cero y menor o igual que 1 es 1, se tiene que el gcd es 1. Por lo que, por Bézout y el desarrollo anterior, se tiene que son primos relativos, ya que su máximo comun divisor es 1.\begin{flushright}$\blacksquare$\end{flushright}

% Pregunta 1.2
		\item \underline{PD} $n$ y $n-1$ son primos relativos para todo $n \geq 2$.\\
		Por inducción se tiene que el caso base es para $n=2$, se traduce en:\\
		\underline{CB} $P(2): \text{gcd}(2, 1) = 1$\\
		Lo que es cierto, ya que el máximo común divisor entre 2 y 1 es efectivamente 1, por lo que son primos relativos.\\
		Ahora la hipotesis de inducción corresponde a asumir el caso enésimo, esto es:\\
		\underline{HI} $P(n): \text{gcd}(n, n-1) = 1$\\
		Y a partir de eso, lo que queda demostrar, es decir, el paso inductivo, corresponde a $P(n) \rightarrow P(n+1)$, esto es:\\
		\underline{PI} $\text{gcd}(n, n-1) = 1 \rightarrow \text{gcd}(n + 1, n) = 1$\\
		Para esto, tomando la hipótesis y operando sobre ella se tiene
		\begin{align*}
			&\text{gcd}(n, n-1) = 1 & \text{Por HI}\\
			&sn + t(n-1) = 1 & \text{Por Bézout}\\
			&sn + tn - t = 1 & \text{Despejando}\\
			&sn + 2tn - tn - t = 1	& \text{Sumando} \ tn \ \text{a ambos lados y despejando}\\
			&(s + 2t)n + (-t)(n + 1) = 1 & \text{Agrupando}\\
			&s'n + t'(n+1) = 1 & \text{Tomando} \ s' = s + 2t \ \text{y} \ t' = -t\\
			&\text{gcd}(n, n+1) = 1 & \text{Por Bézout}
		\end{align*}
		Luego, a partir de la hipótesis de inducción, es posible llegar a lo que se desea demostrar, es decir, que el máximo común divisor entre $n$ y $n+1$ sea 1, por las mismas razones expuestas en \textbf{(1.1)}, considerando que al encontrar la combinación lineal que lleve a 1, este será el gcd, y al ser 1 el gcd, entonces son primos relativos.
		\begin{flushright}$\blacksquare$\end{flushright}

% Pregunta 1.3		
		\item \underline{PD} $a$ y $n$ son primos relativos si, y solo si, $a_0$ y $n$ son primos relativos donde $a_0$ es el digito menos significativo de de la representacion $(a)_n$ en base $n$.\\
		Para esto, se demostrará utilizando igualdades, de modo que no será necesario demostrar ambas direcciones. Antes de eso, se sabe que la representación $(a)_n$ corresponde a:
		$$(a)_n = \sum_{i=0}^{k} a_in^i = a_0 + a_1 n + a_2 n^2 + a_3 n^3 +\ ... \ + a_k n^k$$
		Luego, lo que se quiere demostrar es que:
		$$\text{gcd}(a,n) = 1 \leftrightarrow \text{gcd}(a_0, n) = 1$$
		Entonces, tomando el teorema del algoritmo de euclides, se sabe que $\text{gcd}(a,b)=\text{gcd}(b, a\ \text{mod} \ b)$, para todo $a, b \in \mathbb{Z} - \{0\}$, por lo que:
		$$\text{gcd}(a, n) = \text{gcd}(n, a\ \text{mod} \ n)$$
		Volviendo a lo anterior, se tiene que, utilizando la representación en base $n$ de $a$:
		\begin{align*}
			\text{gcd}(a, n) & = \text{gcd}(n, a\ \text{mod} \ n)& \\
			&= \text{gcd}(n, (a_0 + a_1 n +\ ... \ + a_k n^k)\ \text{mod} \ n) & \text{Sustituyendo}\\
			&= \text{gcd}(n, (a_0\ \text{mod} \ n + a_1 n\ \text{mod} \ n +\ ... \ + a_k n^k\ \text{mod} \ n)\ \text{mod} \ n) & \text{Por distributividad}
		\end{align*}
		Ahora, se sabe que, por distributividad del módulo en el producto, 
		$$a_in^i \ \text{mod} \ n = (a_i\ \text{mod} \ n * n^i \ \text{mod} \ n) \ \text{mod} \ n $$
		Y como $n^i \ \text{mod} \ n = 0, \  \forall i > 0$, se tiene que 
		$$a_in^i \ \text{mod} \ n = (a_i\ \text{mod} \ n * 0) \ \text{mod} \ n = 0$$
		\begin{align*}
			\text{gcd}(a, n)&= \text{gcd}(n, (a_0\ \text{mod} \ n + 0 +\ ... \ + 0)\ \text{mod} \ n) & \text{Por lo anterior}\\
			&= \text{gcd}(n, (a_0\ \text{mod} \ n)\ \text{mod} \ n) & \text{Sumando} \\
			&= \text{gcd}(n, a_0\ \text{mod} \ n) & \text{Revirtiendo o sacando el módulo}\\
			&= \text{gcd}(a_0, n) & \text{Por el teorema, al revés}
		\end{align*}
		De modo que, se demuestra que obtener el gcd entre $a$ y $n$ es exactamente igual a obtener el gcd entre $a_0$ y $n$, por lo que si uno de ellos es 1, el otro también lo será.
		\begin{flushright}$\blacksquare$\end{flushright}
		\end{enumerate}
	\end{pregunta}

	\begin{pregunta}{2}
		\begin{enumerate}
% Pregunta 2.1
		\item ¿Es cierta la afirmación? Demuestre o de un contra-ejemplo, y en caso de dar contra-ejemplo determine una condición adicional para que se cumpla la afirmación, y demuéstrela.\\		
		No es cierta la afirmación ya que, por ejemplo:
		$$18 \equiv 0 \ \text{mod} \ 9 = 6*3 \equiv 0 \ \text{mod} \ 9$$
		Pero,
		$$6 \not\equiv 0 \ \text{mod} \ 9 \ \ \ \text{y} \ \ \ 3 \not\equiv 0 \ \text{mod} \ 9$$
		Ahora, se pide una condición adicional, para que la afirmación se cumpla. Basta con agregar que $n$ sea primo, para que 	se cumpla la afirmación. Hay que demostrar entonces que si $ab \equiv 0 \ \text{mod} \ n$	 y $n$ es primo, entonces $a \equiv 0 \ \text{mod} \ n$ o $b \equiv 0 \ \text{mod} \ n$.
		
		Si $n$ es primo, en particular $a$ y $n$ al igual que $b$ y $n$ son primos relativos entre sí, ya que esto pasa para todos los primos. A menos que sean iguales ($a$ con $n$ o b con $n$ o ambos), pero ese caso es trivial, ya que si son iguales, si bien no habrá inverso, es inmediato que se cumplirá lo pedido, ya que $x = 0 \ \text{mod} \ x$ para todo $x$. Que sean primos relativos, por lo visto en clases, que es justamente el recíproco de lo demostrado en \textbf{(1.1)}, implica que $a$ y $b$ tienen un inverso en $\mathbb{Z}_n$.
		
		Sabiendo lo anterior se tiene que, si $ab \equiv 0 \ \text{mod} \ n$, con $a$ y $b$ primos relativos con $n$, entonces, considerando el inverso de $a$, $a^{-1}$:
		\begin{align*}
			&ab \equiv 0 \ \text{mod} \ n &\\
			&a^{-1}*ab \equiv a^{-1}*0 \ \text{mod} \ n & \text{Multiplicando por el inverso de } a\\
			&b \equiv 0 \ \text{mod} \ n & \text{Ya que } a^{-1}a = 1 \text{, por definición y } a^{-1}0 = 0
		\end{align*}
		Análogamente, considerando el inverso de $b$, $b^{-1}$, se tiene:
		\begin{align*}
			&ab \equiv 0 \ \text{mod} \ n\\
			&ab*b^{-1} \equiv b^{-1}*0 \ \text{mod} \ n\\
			&a \equiv 0 \ \text{mod} \ n
		\end{align*}
		Dichas igualdades solo se cumplen en caso de que $b$ o $a$ sean distinto de $n$, utilizando el inverso, pero de no serlo, como se dijo anteriormente, es el caso trivial.\\
		En ambos casos, a partir de la equivalencia original, utilizando el hecho de que son primos relativos, y por ende tienen inverso, se puede llegar a que, o bien $a \equiv 0 \ \text{mod} \ n$, o $b \equiv 0 \ \text{mod} \ n$, que es exáctamente lo pedido.\\
		Por último, en el caso de que $a$ o $b$ fueran 0, si bien no existiría inverso, la condición se cumple inmediatamente ya que $0 \equiv 0 \ \text{mod} \ n $, para todo $n$.
		\begin{flushright}$\blacksquare$\end{flushright}

% Pregunta 2.2
		\item Encuentre todas las soluciones de $x^2 \equiv 1 \ (\text{mod} \ p)$, con $x \in \mathbb{Z}$ y $p$ primo.
		En primer lugar, para encontrar las soluciones, se puede ver que para todo $n$, $n + 1 \equiv 1 \ \text{mod} \ n$. Por lo que se intentará encontrar soluciones de la forma $\alpha + 1$, donde $\alpha$, sea algún múltiplo o múltiplos de $p$, de modo que su módulo respecto a $p$ sea 0, y solo quede el módulo respecto a 1, que siempre es 1.\\		
		En segundo lugar, se considera que los primos son positivos y parten desde el 2, por lo que la solución que se dará cumplira para esos casos.\\
		Ahora, considerando lo anterior, es fácil ver que $p+1$ es una solución, ya que:
		\begin{align*}
			&(p+1)^2 = p^2 + 2p + 1 & \\
			&(p^2 + 2p + 1) \ \text{mod} \ p & \text{Aplicando módulo}\\
			&(p^2 \ \text{mod} \ p + 2p \ \text{mod} \ p + 1 \ \text{mod} \ p) \ \text{mod} \ p & \text{Por distributividad}\\
			&(0 + 0 + 1 \ \text{mod} \ p) \ \text{mod} \ p & \text{Ya que múltiplos de } p \ \text{mod} \ p \text{ son 0}\\
			&(1) \ \text{mod} \ p & \text{Por distributividad, al revés}
		\end{align*}
		Ahora aplicando lo anterior a la ecuación, se tiene que:
		\begin{align*}
			x^2 \equiv 1 \ (\text{mod} \ p)\\
			(p+1)^2 \equiv 1 \ (\text{mod} \ p)\\
			1 \equiv 1 \ (\text{mod} \ p)
		\end{align*}
		Que cumple con lo pedido, ahora es fácil ver que considerando que cualquier otra propuesta de solución que tenga a 1 como un término libre, y los demás términos como múltiplos de $p$, será solución.
		De ahí que $kp + 1$, con $k \in \mathbb{Z}$ es una solución. Ya que el desarrollo de el cuadrado de dicha solución es $(kp)^2 + 2kp + 1$, que cumple con tener a 1 como término libre, y a los demás términos como múltiplos de p.
		Luego, del mismo modo, al estar considerando un cuadrado, se tiene que $kp - 1$,con $k \in \mathbb{Z}$ también satisface la ecuación, ya que ahora el desarrollo es, $(kp)^2 - 2kp + 1$, que es análogo.
		Dicho lo anterior las soluciones de $x^2 \equiv 1 \ (\text{mod} \ p)$ son:
		\begin{gather*}
			x_1 = kp + 1, \ k \in \mathbb{Z}\\
			x_2 = kp - 1, \ k \in \mathbb{Z}
		\end{gather*}
		
% Pregunta 2.3
		\item \underline{PD} $(n-1)!\equiv (n-1)\ (\text{mod} \ n)$ si, y solo si, $n$ es primo.\\
		Se procede a demostrar ambas direcciones.\\
		($\Rightarrow$) Si $(n-1)!\equiv (n-1)\ (\text{mod} \ n)$\\
		\underline{PD} $n$ es primo.\\
		Asumiendo que $(n-1)!\equiv (n-1)\ (\text{mod} \ n)$, por contradicción también se asume que $n$ es un número compuesto.\\
		Si $n$ es compuesto, tiene al menos un divisor $d$ en $\{2..\sqrt{n}\}$, tal que $d$ divide a $n$. Al $d$ ser menor que $\sqrt{n}$, es también es menor que $n-1$, por lo tanto, tiene que estar contenido en $(n-1)!$ ya que por definición de factorial contiene a todos los números entre 1 y $n-1$. Considerando eso, $d\mid n$ y también, por lo explicado, $d\mid (n-1)!$. En otras palabras, comparten un factor comun.
		
		Ahora operando sobre la hipótesis, se tiene que:
		\begin{align*}
			(n-1)!\equiv (n-1)\ (\text{mod} \ n)\\
			n - (n-1)!\equiv 1 (\text{mod} \ n)
		\end{align*}
		
		Ahora, por la definición de módulo, se tiene que señalar que existe un $q$ tal que $aq+r=b$, es equivalente a $b \equiv r \ (\text{mod} \ a)$. Entonces, en lo anterior, se tiene que:
		\begin{align*}
			& nq + 1 = n - (n-1)! & \text{Con } r=1, \ b=n - (n-1)! \ \text{y } a=n\\
			& 1 = n - nq - (n-1)! & \text{Despejando}\\
			& 1 = (1-q)n + (-1)(n-1)! & \text{Agrupando}
		\end{align*}
		De lo que, con $s=1-q$ y $t=-1$, se llega a que el gcd$(n, (n-1)!)=1$, por lo visto en las demostraciones de \textbf{1}, es decir por Bézout, de modo que son primos relativos. He aquí la contradicción ya que si son primos relativos, no pueden tener ningún factor en comun, y si $n$ no es primo, como se dijo antes, si tiene al menos un factor en comun $d$ que divide a $n$ y a $(n-1)!$.\\
		
		($\Leftarrow$) Si $n$ es primo\\
		\underline{PD} $(n-1)!\equiv (n-1)\ (\text{mod} \ n)$\\
		Asumir que $n$ es primo, implica que en $\mathbb{Z}_n$ no existirán factores primos de $n$, y que $n$ es primo relativo con todos los elementos de $\mathbb{Z}_n$. Así, al ser primos relativos, por lo visto en clases, es decir, el recíproco de lo demostrado en \textbf{(1.1)}, cada elemento tendrá un inverso en $\mathbb{Z}_n$, de modo que para cada $a \in \mathbb{Z}_n$, se tiene que $a^{-1}a\equiv 1 \ (\text{mod} \ n)$ con $a^{-1} \in \mathbb{Z}_n$ igualmente.\\
		Ahora por propiedad de los módulos, se tiene que si $a \equiv b \ (\text{mod} \ n)$ y $c \equiv d\ (\text{mod} \ n)$, entonces se tiene $ac \equiv bd \ (\text{mod} \ n)$, de modo que se pueden multiplicar todas las expresiones necesarias, de modo de formar $1*2*3...*(n-2)$. Esto porque $n-1$ tiene por inverso a si mismo, por lo tanto no se considera en dicho producto, porque se repetirían factores.\\
		Entonces, se tiene, por la propiedad anterior:
		$$1*2*3...*(n-2)\equiv 1 \ (\text{mod} \ n)$$
		Ahora, multiplicando a ambos lados por $n-1$ se tiene que:
		$$1*2*3...*(n-2)*(n-1)\equiv (n-1) \ (\text{mod} \ n)$$
		$$(n-1)!\equiv (n-1) \ (\text{mod} \ n)$$
		Que es justamente lo que se intentaba demostrar.
		\begin{flushright}$\blacksquare$\end{flushright}
		\end{enumerate}

	\end{pregunta}

\end{document}
